%% ------------------------------------------------------------
%% PACKAGES
%% ------------------------------------------------------------

%% For \circledast
\usepackage{amssymb,amsfonts,amsmath}

\usepackage{algorithm}
%\usepackage{algorithmic}
\usepackage[noend]{algpseudocode}

%% For \mathscr
\usepackage[mathscr]{eucal}

%% For \llbracket and \rrbracket, \varoast, \varoslash
\usepackage{stmaryrd}

%% For \boldsymbol
\usepackage{amsbsy}

%% For \bm (bold math)
\usepackage{bm}

%% For \set, \Set
\usepackage{braket}

%% For \multirow
\usepackage{multirow}

%% For "X" columntype that automatically calculates width
%\usepackage{tabularx}
%\newcolumntype{Y}{>{\raggedright\arraybackslash}X}

%% For \sfrac
\usepackage{xfrac}

%% For attractive boxed equations
%% https://tex.stackexchange.com/questions/20575/attractive-boxed-equations
%usepackage{empheq}
%usepackage[most]{tcolorbox}

%% For special macros
\usepackage{xparse}

%% For special environments
\usepackage{environ}

%% For straight quote in verbatim 
\usepackage{upquote}

%% For boxed verbatim
\usepackage{moreverb}

%% For \FloatBarrier
\usepackage{placeins}

%% For pretty lists
\usepackage{enumitem}

%% Tikz - For making pretty pictures
\usepackage{tikz}
\usetikzlibrary{calc}
% \usetikzlibrary{3d}
% \usetikzlibrary{patterns}
% \usetikzlibrary{arrows}
% \usetikzlibrary{decorations.pathreplacing}

%% For plots
\usepackage{pgfplots}

%% Subfloats
\usepackage[font=footnotesize,justification=Centering,singlelinecheck=false]{subfig}

%% Redundant but causes emacs to do useful things
\usepackage{cleveref}

%% For trimming the PNG images by px
%\pdfpxdimen=\dimexpr 1 in/96\relax

\usepackage{setspace}

%% ------------------------------------------------------------
%% MACROS - USING XPARSE METHODS FOR DEFINITION
%% ------------------------------------------------------------

\definecolor{grey}{RGB}{100,100,100}
%KDD\algrenewcommand{\algorithmiccomment}[1]{\hfill\textcolor{grey}{// #1}}

\newenvironment{inlinemath}{$}{$}

% Quad-text-quad
\NewDocumentCommand \qtext {m} {\quad\text{#1}\quad}

% Reals
\NewDocumentCommand \Real {} {\mathbb{R}}

% Natural numbers
\NewDocumentCommand \Natural {} {\mathbb{N}}

% Binary numbers
\NewDocumentCommand \Binary {} {\set{0,1}}

% Range {1,2,\dots, n}
\NewDocumentCommand \Range { m } { \set{1,2,\dots,#1} }

% Tensor
\NewDocumentCommand \T { O{} m } {\boldsymbol{#1\mathscr{\MakeUppercase{#2}}}}

% Matrix
% (Can't use \M because it denotes the low-rank "model" tensor.)
\NewDocumentCommand \Mx { O{} m } {{\bm{#1\mathbf{\MakeUppercase{#2}}}}} 

% Vector
\NewDocumentCommand \V { O{} m } {{\bm{#1\mathbf{\MakeLowercase{#2}}}}}

% --- Data Tensor: X ---

% Tensor X
\NewDocumentCommand \X { } {\T{X}}

% Number of nonzeros in X
\NewDocumentCommand \nzX {} {\nnz{\X}}

% Mode-k Unfolding of tensor X
\NewDocumentCommand \Xk { G{k} } {\Mx{X}_{(#1)}}

% Single element of tensor X
\NewDocumentCommand \xe { } {x_{i}}

% --- Model Tensor: M ---

% Tensor Model M
\NewDocumentCommand \M {} {\T{M}}
\NewDocumentCommand \Mtrue {} {\M_{\rm true}}
% Tensor Model Single Element
\NewDocumentCommand \me { } {m_{i}}

% --- Function ---

% Function F wrt X and M
\NewDocumentCommand \FXM {} {F(\X,\M)}

% Function f wrt x and m, ' means derivative wrt m
\NewDocumentCommand \fxm {t' G{x} G{m}} {\IfBooleanTF{#1}{\FPD{f}{m}}{f}(#2,#3)}

% Function f wrt x_i and m_i, ' means derivative wrt m
\NewDocumentCommand \fxme {t'} {\IfBooleanTF{#1}{\fxm'{\xe}{\me}}{\fxm{\xe}{\me}}}

% --- Weight Tensor: W ---

% Tensor W
\NewDocumentCommand \W {} {\T{W}}

% % Mode-k Unfolding of Tensor W
% %\NewDocumentCommand \Wk {} {\Mx{W}_{(k)}}

% Tensor W Single Element
\NewDocumentCommand \we {} {w_i}
% {
%   \IfBooleanTF{#1}
%   {w_{\xi}}
%   {w_{i}}
% }

% --- Partial Gradient Tensor and Stochastic Version: Y ---

% Tensor Y
\NewDocumentCommand \Y {} {\T{Y}}
\NewDocumentCommand \Ys {} {\T[\tilde]{Y}}

% Mode-k Unfolding of Tensor Y
\NewDocumentCommand \Yk { O{k} } {\Mx{Y}_{(#1)}}
\NewDocumentCommand \Yks { O{k} } {\Mx[\tilde]{Y}_{(#1)}} %stochastic

% Tensor Y Single Element
\NewDocumentCommand \ye { } {y_{i}}
\NewDocumentCommand \yes { } {\tilde y_i}

% Matricized Y Single Element
\NewDocumentCommand \yke { } { y_{(k)} (i_k,i_k') }

% --- Factor Matrices: A_k ---

% K-th Factor Matrix: A^(k) (trailing ' for transpose)
% \Ak{1} changes the superscript to 1
% \Ak{1} does both (quote must come after the braces)
\NewDocumentCommand \Ak { G{k} t' t"  } { \Mx{A}_{#1}\IfBooleanTF{#2}{^{\intercal}}{}\IfBooleanTF{#3}{^{\phantom{\intercal}}}{} }
\NewDocumentCommand \EstAk {} {\Mx[\hat]{A}_k}
\NewDocumentCommand \Akj {O{k} G{j}} {\V{a}_{#1}(:,#2)}
\NewDocumentCommand \EstAkj {O{k} G{j}} {\V[\hat]{a}_{#1}(:,\pi(#2))}

\NewDocumentCommand \Akset { } {\set{\Ak }}
\NewDocumentCommand \Bkset { } {\set{\Bk }}
\NewDocumentCommand \Ckset { } {\set{\Ck }}

%\NewDocumentCommand \AkAkt { G{k} } {\Ak{#1}'\Ak{#1}"}

% Factor Matrix Element a^(k)_{i_k j}
% \NewDocumentCommand \Ake { G{k} G{i} G{j} } {
%   a_{#1}(#2_{#1},#3)
% }

% Stacked Ak's
%\NewDocumentCommand \avec {} {\V{a}}

% --- Gradient wrt Factor Matrices: G_k (+ Stochastic Versions)---

\NewDocumentCommand \Gk { G{k}  } { \Mx{G}_{#1} }
\NewDocumentCommand \Gks { G{k}  } { \Mx[\tilde]{G}_{#1} } % stochastic version
\NewDocumentCommand \Gkset { } {\set{\Gk}}
\NewDocumentCommand \Gksset { } {\set{\Gks}}

% --- Ktensor format ---

% Ktensor (* = weights)
\NewDocumentCommand \KT { s } {
  \llbracket
  \IfBooleanTF{#1}{\lvec;}{}
  \Ak{1}, \Ak{2}, \dots,  \Ak{d} \rrbracket
}

% Matrices used in ADAM 
\NewDocumentCommand \Bk { G{k} s } { \IfBooleanTF{#2}{\Mx[\hat]{B}_{#1}}{\Mx{B}_{#1}} }
\NewDocumentCommand \Ck { G{k} s } { \IfBooleanTF{#2}{\Mx[\hat]{C}_{#1}}{\Mx{C}_{#1}} }

% --- Khatri-Rao Products of Factor Matrices: Z ---

% Khatri-Rao of all Factor Matrices but K-th : Z^(k)
\NewDocumentCommand \Zk { G{k} t' t"} {\Mx{Z}_{#1}\IfBooleanTF{#2}{^{\intercal}}{}%
  \IfBooleanTF{#3}{^{\phantom{\intercal}}}{}}

% Khatri-Rao of all Factor Matrices : Z
%\NewDocumentCommand \zvec { } {\Mx{Z}}

\NewDocumentCommand \ZkZkt { G{k} } {\Zk{#1}'\Zk{#1}"}

% Z Single Element
%\DeclareDocumentCommand \zke { } { z_{k} (i_k',j) }

% --- Sampled Indices ---
%\NewDocumentCommand \SI {} {{\tilde{\mathcal{S}}}}
\NewDocumentCommand \SIcnt {} {\tilde{s}_i}
\NewDocumentCommand \NIcnt {} {\tilde{p}_i}
\NewDocumentCommand \ZIcnt {} {\tilde{q}_i}

% --- counts and probabilities for zeros and nonzeros

% zero sample size
\NewDocumentCommand \szero {} {s_{\text{\tiny zero}}}
\NewDocumentCommand \sreject {} {s_{\text{\tiny reject}}}

% zero probability
\NewDocumentCommand \pzero {} {p_{\text{\tiny zero}}}
% nonzero probability
\NewDocumentCommand \pnz {} {p_{\text{\tiny nonzero}}}

% --- Important Sets ---

% The set of all indices
%\NewDocumentCommand \I {} {\mathcal{I}}
%\NewDocumentCommand \Gsamp {} {\Omega}
%\NewDocumentCommand \Gweights {} {\set{ \omega_i }_{ i \in \Gsamp}}
%\NewDocumentCommand \Fsamp {} {\T[\tilde]{V}}
\NewDocumentCommand \Fs {} {\T[\tilde]{V}}
\NewDocumentCommand \fse {} {\tilde v_i}
% \NewDocumentCommand \Iskik {} {{\widetilde{\mathcal{I}}}_k(i_k)}
\NewDocumentCommand \Fweights {} {\set{ \psi_i }_{i \in \Fsamp}}
%\NewDocumentCommand \Sizes {} {\set{ n_k }}

% Nonzeros & Zeros
%\newcommand\Ntilde{\stackrel{\sim}{\smash{\mathcal{N}}\rule{0pt}{1.1ex}}}
%\newcommand\Ntilde{{\widetilde{\mathcal{N}}}}
%\NewDocumentCommand \Nzs { s } {\IfBooleanTF{#1}{\Ntilde}{\mathcal{N}}}
%\NewDocumentCommand \Zs {} {\mathcal{Z}}
% \NewDocumentCommand \Nkik { s } {\IfBooleanTF{#1}{\Ntilde}{\mathcal{N}}_k(i_k)}
% \NewDocumentCommand \Zkik {} {\mathcal{Z}_k(i_k)}
%\NewDocumentCommand \etakik { s } {\IfBooleanTF{#1}{\tilde}{}\eta_k(i_k) }
% \NewDocumentCommand \zetakik {} {\zeta_k(i_k) }

% --- Some useful matrices

%\NewDocumentCommand \A { } {\Mx{A}}
%\NewDocumentCommand \B {t' } {\Mx{B}\IfBooleanTF{#1}{^{\intercal}}{}}

% --- Other Stuff ---


% Set of natural numbers {1,2,...,#2}
% \NewDocumentCommand \nnset {s m}
% {
%   \IfBooleanTF{#1}
%   {\set{1,2,\dots, #2}}
%   {\set{1, \dots, #2}}
% }

% First Partial Derivative (* for in-line)
\NewDocumentCommand \FPD { s m m } {
  \IfBooleanTF{#1}
  {\tfrac{\partial #2}{\partial #3}}
  {\frac{\partial #2}{\partial #3}}
}

% The PDF
%\NewDocumentCommand \pdf {m m} {p(#1\,\vert\,#2)}

\NewDocumentCommand{\iid}{}{i.i.d.\@ }
%\DeclareMathOperator*{\minimize}{minimize}
\DeclareMathOperator{\diag}{diag}
%\DeclareMathOperator{\trace}{tr}
\DeclareMathOperator{\rank}{rank}
%\NewDocumentCommand{\vc}{}{\textsc{vec}}
%\DeclareMathOperator{\logit}{logit}
%\DeclareMathOperator{\expit}{expit}

% Expectation
\NewDocumentCommand{\Exp}{m}{\mathbb{E}[#1]}

% Number of nonzeros
\NewDocumentCommand{\nnz}{m}{\text{nnz}(#1)}

\NewDocumentCommand{\LineFor}{m m}{%
  \State\textbf{for} {#1}, \textbf{do} {#2}, \textbf{end}
  }


\NewDocumentCommand \plow {} {\rho_{\rm low}}
\NewDocumentCommand \phigh {} {\rho_{\rm high}}
  
% \DeclareMathOperator{\prob}{Pr}
%\DeclareMathOperator{\Prob}{Pr}
%\DeclareMathOperator{\var}{Var}
%\NewDocumentCommand{\Bern}{}{{\rm{Bernoulli}}}
%\NewDocumentCommand{\ifrac}{s m m}{#2 \, \big/ \, \IfBooleanTF{#1}{#3}{(#3)}}
%\NewDocumentCommand{\logfrac}{m m}{\log \bigl(\, #1 \, \big/ \, (#2) \, \bigr) }

% --- Notes to each other ---
\NewDocumentCommand \Note { m } {\textcolor{red}{#1}}

% \NewDocumentCommand \bk { G{k}  } { \V{u}_{#1} }
% \NewDocumentCommand \ck { G{k}  } { \V{v}_{#1} }
% \NewDocumentCommand \ve {} {\V{e}}
% \NewDocumentCommand \Bk {G{k}} {\Mx{U}_{#1}}
% \NewDocumentCommand \Ck {G{k}} {\Mx{V}_{#1}}
% \NewDocumentCommand \dOmega {} {\delta_{i \in \Omega}}
% \NewDocumentCommand \dPsi {} {\delta_{i \in \Psi}}

% \NewDocumentCommand \OmegaN {} {\Omega_{\textrm{n}}}
% \NewDocumentCommand \OmegaZ {} {\Omega_{\textrm{z}}}
% \NewDocumentCommand \PsiZ {} {\Psi_{\textrm{z}}}
% \NewDocumentCommand \sN {} {s_{\textrm{n}}}
% \NewDocumentCommand \sZ {} {s_{\textrm{z}}}
% \NewDocumentCommand \cN {} {c_{\textrm{n}}}
% \NewDocumentCommand \cZ {} {c_{\textrm{z}}}

% Suppress hyphenation
\hyphenation{MTTKRP}




% % From https://tex.stackexchange.com/questions/155981/how-to-make-for-all-xxx-do-appear-on-one-line?noredirect=1&lq=1
% \makeatletter
% \newcommand{\LineFor}[3][default]{%
%   \ALC@it\algorithmicfor\ #2\ \algorithmicdo%
%   \ALC@com{#1}\ #3%
% }
% \newcommand{\LineEndFor}{\ALC@it\algorithmicendfor}
% \makeatother

% From https://tex.stackexchange.com/questions/51019/how-can-i-put-a-curly-brace-inside-an-algorithm-to-group-code-lines
\newcommand{\tikzmark}[1]{\tikz[overlay,remember picture] \node (#1) {};}

\newcommand*{\AddNote}[5]{%
    \begin{tikzpicture}[overlay, remember picture]
        \draw [decoration={brace,amplitude=0.5em},decorate,grey]
            ($(#3)!(#1.north)!($(#3)-(0,1)$)$) --
            ($(#3)!(#2.south)!($(#3)-(0,1)$)$)
            node [align=center, text width=#4, pos=0.5, anchor=west, right=1em] {//
              #5};
    \end{tikzpicture}
}%

\newcommand{\rundesc}{Each dashed line represents a single run, and the markers signify epochs. The marker is an asterisk  if the true solution was recovered and a dot otherwise. Solid lines represent the median. Dashed black line is the function value estimate for the true solution.}

\newcommand{\boxplotdesc}{}

\newcommand{\lossdesc}{The \emph{same} set of samples is used to estimate the loss across every individual run.}

% % #1: Label/Filename
% % #2: Description
% % #3: Extra sentence
% \NewDocumentCommand{\samplefig}{m m m}{
% \begin{figure}
%   \centering
%   \subfloat[Individual runs. \rundesc\@ \lossdesc]%
%   {\label{fig:#1-sample-size-runs}~~~~\includegraphics[trim=65 0 75 0, clip, scale=0.45]%
%     {fig-#1-sample-size.png}~~~~~}\\
%   \subfloat[Number of times the true solution was recovered, i.e., cosine similarity $\geq$ 0.9, for each number of gradient samples.]{
%   \label{fig:#1-sample-size-recoveries}~~~~\includegraphics[scale=0.45]{fig-#1-sample-size-bar.png}~~}~~
%   \subfloat[Box plot of mean time per epoch and mean time per sample. \boxplotdesc]{
%   \label{fig:#1-sample-size-epoch-time}\includegraphics[scale=0.45]{fig-#1-sample-size-time.png}}  
% \caption{Investigating the effect of stochastic gradient sample size by running GCP-OPT-ADAM #2, with a variable number of samples per stochastic gradient.
%   For each instance, we do 25 runs with different initial guesses. (The same 25 initial guesses are used for each instance.)
%      #3 %If the function value increases, the step length is reduced by a factor of ten and the method continues with the solution from the prior epoch.
%   }
%   \label{fig:#1-sample-size}
% \end{figure} 
% }

\NewDocumentCommand{\cpic}{m}{\includegraphics[trim=0 0 10 0, clip, scale=0.4]{fig-chicago-details-#1}}
\NewDocumentCommand{\comppic}{m}{\label{fig:comp-#1}\includegraphics[trim=0 0 10 0, clip, scale=0.4]{fig-chicago-semistrat-#1}}
\NewDocumentCommand{\morecomppic}{m}{%
  \begin{figure}%
    \includegraphics[trim=0 0 10 0, clip, scale=0.4]{fig-chicago-semistrat-#1}
    \caption{Component #1 for Chicago crime tensor using semi-stratified sampling.}
    \label{fig:comp-#1}%
  \end{figure}%
}

\newcommand{\datafile}{}

\newcommand{\GB}[1]{\textcolor{red}{\textbf{GB}: #1}}
\newcommand{\KDD}[1]{\textcolor{blue}{\textbf{KDD}: #1}}
  
%%% Local Variables:
%%% mode: latex
%%% TeX-master: "gcp_sparse"
%%% End:
