GentenMPI is a toolkit of sparse canonical polyadic (CP)
tensor decomposition algorithms that is
designed to run effectively on distributed-memory high-performance computers.
Its use of distributed-memory parallelism enables it to efficiently 
decompose tensors that are too large for a single compute node's memory.
GentenMPI leverages Sandia's decades-long investment in the Trilinos solver
framework for much of its parallel-computation capability.  Trilinos contains
numerical algorithms and linear algebra classes that have been optimized for
parallel simulation of complex physical phenomena.  This work applies these 
tools to the data science problem of sparse tensor decomposition.  In this
report, we describe the use of Trilinos in GentenMPI, extensions needed 
for sparse tensor decomposition, and implementations of the CP-ALS
(CP via alternating least squares~\cite{CC70,Harshman70}) and 
GCP-SGD (generalized CP via 
stochastic gradient descent~\cite{HKD18, HoKoDu20, KH19})
sparse tensor decomposition algorithms.  We show that GentenMPI can 
decompose sparse tensors of extreme size, e.g., a 12.6-terabyte
tensor on 8192 computer cores.  We demonstrate that the Trilinos backbone 
provides good strong and weak scaling of the tensor decomposition algorithms.
